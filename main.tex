\documentclass[a4paper,10pt]{article}
\usepackage[top=0.40in, bottom=0.50in, left=0.50in, right=0.70in]{geometry}

% For mathematical symbols
\usepackage{amsmath}
\usepackage{graphicx}
\usepackage{array}

%For adding links
\usepackage{hyperref}
\hypersetup{
    colorlinks=true,
    linkcolor=blue,
    filecolor=magenta,      
    urlcolor=cyan,
}

%format section titles
\usepackage{titlesec}
\titleformat{\section}
{\large \bfseries}
{}
{0em}       % 0em removed section numbers
{}[\titlerule]  % underline every section

%Reduce space between lines
\renewcommand{\baselinestretch}{0.9}

%Use for page count in footer ex "2 of 10"
\usepackage{lastpage}
\usepackage{fancyhdr}
\pagestyle{fancy}
\fancyhead{}
\fancyfoot{}
\renewcommand{\headrulewidth}{0pt}
\renewcommand{\footrulewidth}{0pt}
\fancyfoot[C]{\thepage\ of \pageref{LastPage}}

\begin{document}
% Resume header
\begin{table}
    \begin{tabular}{m{4cm} m{8cm} m{3cm}}
         \includegraphics[scale=0.2]{nit} & 
         \begin{tabular}{l}
         \textbf{Prashant Singh}  \\
         \textbf{Operations Research} \\ \textbf{National Institute of Technology, Durgapur} \\
         \end{tabular}
         & 
         \begin{tabular}{l} \textbf{M.Tech.} \\ \textbf{prashantexploring@gmail.com} \\
         \textbf{7676798841} \\
         \href{https://www.linkedin.com/in/techyprashant/}{LinkedIn Profile}
         \end{tabular}
    \end{tabular}
\end{table}
    
    
	% Educational Details
	\section{Education}
	\begin{itemize}
		\item M.Tech., National Institute of Technology, Durgapur
		\begin{itemize}
			\item M.Tech. in Operations Research,  CGPA 7.95, (2016 - 2018)
		\item Graduate Courses : Programming Language and Data Structure, Automata and Algorithms, Discrete Mathematics, Operations research,  Probability and Statistics, Optimization Techniques, Numerical Methods
		\end{itemize}
		
		\item B.Tech., National Institute of Technology, Durgapur
		\begin{itemize}
			\item B.Tech. in Biotechnology with open electives from Computer Science,  CGPA 7.0, (2007 - 2011)
		\item Undergraduate Courses : Data Structures, Computer Networks, Operating Systems, Communication Network, Database Management Systems
		\end{itemize}
	\end{itemize}
	
	% Different section of resume starts from here
	\section{Employment(s)}
	\begin{itemize}
	        \item Software Developer, VHR Solutions (Feb 2020 - Sep 2020)
                \begin{itemize}
                    \item Worked in payments team of Infosys’s Finacle product . As a team member I was involved in implementing the code for handling the Bank Identifier Code . Written a Java JDBC code for exporting table rows from Oracle database . The exported rows are in form of insert query and are suitable for Postgres database . Have written a bash script for execution of multiple sql files on remote postgres database .
                \end{itemize}
			\item Teaching Assistant, M.Tech. NIT Durgapur (July 2017 - May 2018)
                \begin{itemize}
                    \item Numerical Analysis Lab using C programming, (10 students) May 2018 .
                    \item Operations Research Lab using Matlab, (10 students)Dec 2017 .
                \end{itemize}
            \item System Engineer, TCS(Tata Consultancy Services, Chennai) (July 2011 - July 2013)
		        \begin{itemize}
					\item Created web services to migrate data from legacy system IBM DB2 database to latest oracle database using Oracle Middleware Technology .
					\item Monitored and managed integration servers installed on Redhat systems using linux commands and script.
					The data  of DB2 was exposed in XML  and the web services were consuming these data and inserting it into Oracle database .
			\end{itemize}
	\end{itemize}

	\section{Projects}
	
	\begin{itemize}
		\item \textbf{Question Answer AI Bot} , Oct 2020 
		
		Created a standalone AI Bot to answer questions . Input was few wikipedia pages as text files . Used TF-IDF technique to give most relevant answer to the given user query .
		\href{https://github.com/eelectron/questions}{project link}.  
	\end{itemize}
	
	\begin{itemize}
	\item \textbf{Pizza Orders Website} , Jan 2020
	
	 Created a \href{https://pizzacs50w.herokuapp.com/}{web application} for handling pizza restaurant's online orders , using Django web framework and currently deployed on Heroku platform: \href{https://github.com/eelectron/pizza}{project link}
				
				
					\begin{itemize}
						\item Features : User can register, login, add and save item to cart, get order confirmation email
					\end{itemize}
						
				\item \textbf{Flack, Chatting Web App} , Jan 2020 
				
				Created a chatting application named \href{https://flackthechat.herokuapp.com}{Flack} , similar to Slack , using python, flask and socketIO . It is currently deployed on Heroku . User can register, create new channel , delete their post, check others post in real time using socketIO . : \href{https://github.com/eelectron/flack}{project link} 
				
				\item \textbf{Book Review Website} , Dec 2019
				
				Created a book review website, \href{http://lookthebook.herokuapp.com}{lookthebook} . It is currently deployed on Heroku .Technology used are Html, Javascript, Python, Flask, Postgresql database : \href{https://github.com/eelectron/BookReview}{project link} 
				\begin{itemize}
					\item Features : User can register, login and logout, give comment, serach book, view ratings of book 
				\end{itemize}
				
				\item M.Tech project: Recognition of printed Odia(A language spoken in India) character using neural network , June 2018.
				\begin{itemize}
				    \item recognized 62 basic Odia characters (10 numerical, 12 vowels, 40 consonants) in a given image and convert them to computer editable text using deep neural network. Got an accuracy of 97.75
				    \item Input layers = 784, hidden layers = 300, output layers = 62
				    \item Technology: Python, OpenCV
				\end{itemize}
					
                			
				\item \textbf{Game Development} 
				    \begin{itemize}
				    \item 
				    Created web based game \href{https://gitlab.com/psonlinux/2048-game-web}{2048} and also in command line form , in C++, \href{https://gitlab.com/psonlinux/2048}{code} .
				        \item \href{https://en.wikipedia.org/wiki/15_puzzle}{"Game of Fifteen"} in C, \href{https://gitlab.com/psonlinux/game-of-fifteen}{code}
\item Breakout in C, \href{https://gitlab.com/psonlinux/breakout}{code}				    
				    \end{itemize}
			\end{itemize}
			
    \section{Additional Courses}
	\begin{itemize}
	        \item Deep Learning Specialization, Coursera, Jan 2019
            \item CS50's Introduction to Computer Science, HarvardX, July 2019
	        \item \href{https://credentials.edx.org/credentials/bc6063317f6249019179899eb0c5aacb/}{Computational Thinking using Python}, MITx, Aug, 2018
	        \item Machine Learning Stanford University, Coursera, November 2017, Grade Achieved: 97.6\%
	        \item Algorithms: Design and Analysis, Part 1, Coursera, Grade Achieved: 84.5\%, September 2015
	    \end{itemize}
	
	\section{Skills and Technologies}
	\begin{itemize}
		\item Web Framework : Django, Flask, JavaEE
		\item Algorithms , Data Structure, Object Oriented Design, Responsive Web Design
        		\item Programming languages: C++, Python, Java , C , Matlab 
		\item Frontend : Ajax, Javascript, Html, CSS
		\item Backend : Postgres, Oracle, Sqlite
	\end{itemize}
	
	\section{Achievements}
	\begin{itemize}
	\item \href{http://files.codingninjas.in/certificate120385127582f078b1b89fee75fd48d65c87741.pdf}{Certificate} of excellence in Competitive Programming course on Coding Ninjas
		\item All India Rank 1593 among 1,08,495  in \href{https://drive.google.com/file/d/0B-2PYJZE99wQb1NCVHZFQjBsb0U/view}{GATE-2016} in Computer Science .
                \item Cummins Inc.’s appreciation certificate \href{https://drive.google.com/file/d/0B2kfGzkUmzWdbHJJZTRSLWJRVEE/view}{certificate} .
	\end{itemize}
\end{document}

